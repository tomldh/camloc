\documentclass{article}
\usepackage[utf8]{inputenc}
\usepackage{float}
\usepackage{graphicx}
\usepackage{geometry}
\geometry{margin=1.2in}
\usepackage{amsmath}
\usepackage{cool}

\makeatletter
\renewcommand*\env@matrix[1][\arraystretch]{%
  \edef\arraystretch{#1}%
  \hskip -\arraycolsep
  \let\@ifnextchar\new@ifnextchar
  \array{*\c@MaxMatrixCols c}}
\makeatother

\begin{document}

\section{Calculation of Analytical Gradient in Kabsch Algorithm}

$X$ is a $N \times 3$ matrix representing a set of $N$ scene points in 3D space. The $N\times3$ matrix $P$ is the corresponding measurements of the $N$ scene points. $x_{ij}$ and $p_{ij}$ represent the $(i,j)$ element of the matrix $X$ and $P$ respectively. \par

Let $\textbf{t} \in \mathbb{R}^3$ represent the translation of the scene points in $X$ so that its centroid coincides with the origin of the coordinate system. The centroid of the dataset is defined as the average position of all the points, therefore, each component $t_k$, $k\in[1,3]$ of the translation vector is computed as follows:

\[t_k = -\frac{1}{N}\sum_{i=1}^{N}x_{ij}, \quad \forall j=k\]


The partial derivatives of $\textbf{t}$ w.r.t the coordinates of the scene points are defined as follows:  

\[ \pderiv[1]{t_k}{x_{ij}} =  
\begin{cases} 
      -\frac{1}{N} & ,if \quad k = j \\
      0 & ,if \quad k \neq j 
\end{cases}
\]

Let $\tilde{X} = X + \textbf{t}$, and now the centroid of $\tilde{X}$ coincides with the origin. Assuming the centroid of $P$ has been overlapped with the origin, there exists a $3\times3$ rotation matrix $R$ such that $P = R\tilde{X}$. To find out the optimal rotation matrix $R$, perform the following Kabsch algorithm: \par
\begin{enumerate}
\item Compute the $3\times3$ covariance matrix $A = P^T\tilde{X}$.
\item Perform SVD of the matrix $A = USV^T$.
\item $d = det (UV^T)$.
\item $R = U
\begin{pmatrix}
1 & 0 & 0 \\
0 & 1 & 0 \\
0 & 0 & d 
\end{pmatrix}
V^T $, where $d=\pm1$.
\end{enumerate}

How to calcuate $\pderiv[1]{R}{x_{ij}}$ ? \par

\begin{enumerate}
\item Compute $\pderiv[1]{A}{x_{ij}}$.
\item Compute $\pderiv[1]{U}{a_{kl}}$, $\pderiv[1]{D}{a_{kl}}$ and $\pderiv[1]{V^T}{a_{kl}}$ for each element $a_{kl}$ of matrix $A$, according to method proposed by Papadopoulo \emph{et al.}
\item $R = U
\begin{pmatrix}
1 & 0 & 0 \\
0 & 1 & 0 \\
0 & 0 & d
\end{pmatrix}
V^T$ and $\pderiv[1]{R}{a_{kl}} = \pderiv[1]{U}{a_{kl}}
\begin{pmatrix}
1 & 0 & 0 \\
0 & 1 & 0 \\
0 & 0 & d
\end{pmatrix}
V^T + U
\begin{pmatrix}
1 & 0 & 0 \\
0 & 1 & 0 \\
0 & 0 & d
\end{pmatrix}
\pderiv[1]{V^T}{a_{kl}}$, where $d=\pm1$.
\item $\pderiv[1]{R}{x_{ij}} = \pderiv[1]{R}{a_{kl}} \times \pderiv[1]{a_{kl}}{x_{ij}}$
\end{enumerate}
Remarks:
\begin{enumerate}
\item $x_{ij}$ and $p_{ij}$ should be correspondence.
\item  $\pderiv[1]{A}{x_{ij}}$ will have its element $\pderiv[1]{a_{kl}}{x_{ij}}=0$ if $l \neq j$ (See example below). Therefore, $\pderiv[1]{a_{kl}}{x_{ij}}$ needs to be chosen carefully for use in calculating $\pderiv[1]{R}{x_{ij}}$, i.e. with $l=j$.
\end{enumerate}
\bigskip
Example. \par
For $N=4$, 
\[
X =
\begin{pmatrix}

x_{11} & x_{12} & x_{13} \\
x_{21} & x_{22} & x_{23} \\
x_{31} & x_{32} & x_{33} \\
x_{41} & x_{42} & x_{43}

\end{pmatrix}
\quad
P = 
\begin{pmatrix}

p_{11} & p_{12} & p_{13} \\
p_{21} & p_{22} & p_{23} \\
p_{31} & p_{32} & p_{33} \\
p_{41} & p_{42} & p_{43}
\end{pmatrix}
\]

The translation vector,
\[
\textbf{t} = 
\begin{pmatrix}
t_1 \\
t_2 \\
t_3 
\end{pmatrix}
=
\begin{pmatrix}[1.5]

-\frac{1}{4}(x_{11}+x_{21}+x_{31}+x_{41}) \\
-\frac{1}{4}(x_{12}+x_{22}+x_{32}+x_{42}) \\
-\frac{1}{4}(x_{13}+x_{23}+x_{33}+x_{43})

\end{pmatrix}
\]

e.g. the partial derivatives of $\textbf{t}$ w.r.t to $x_{43}$ is
\[
\pderiv[1]{\textbf{t}}{x_{43}} = 
\begin{pmatrix}[1.5]

\pderiv[1]{t_1}{x_{43}} \\
\pderiv[1]{t_2}{x_{43}} \\
\pderiv[1]{t_3}{x_{43}}

\end{pmatrix}
=
\begin{pmatrix}[1.5]
0 \\
0 \\
-\frac{1}{4}

\end{pmatrix}
\]

Apply the translation to $X$,

\[
\tilde{X} = X + \textbf{t}^T =
\begin{pmatrix}

x_{11}+t_1 & x_{12}+t_2 & x_{13}+t_3 \\
x_{21}+t_1 & x_{22}+t_2 & x_{23}+t_3 \\
x_{31}+t_1 & x_{32}+t_2 & x_{33}+t_3 \\
x_{41}+t_1 & x_{42}+t_2 & x_{43}+t_3

\end{pmatrix}
\]

\[
\pderiv[1]{\tilde{X}}{x_{43}} = \pderiv[1]{X}{x_{43}} + \pderiv[1]{\textbf{t}^T}{x_{43}} = 
\begin{pmatrix}
0 & 0 & 0 \\
0 & 0 & 0 \\
0 & 0 & 0 \\
0 & 0 & 1 
\end{pmatrix}
+
\begin{pmatrix}
0 & 0 & -\frac{1}{4} \\
0 & 0 & -\frac{1}{4} \\
0 & 0 & -\frac{1}{4} \\
0 & 0 & -\frac{1}{4}
\end{pmatrix}
=
\begin{pmatrix}
0 & 0 & -\frac{1}{4} \\
0 & 0 & -\frac{1}{4} \\
0 & 0 & -\frac{1}{4} \\
0 & 0 & \frac{3}{4}
\end{pmatrix}
\]

The covariance matrix $A=P^T\tilde{X}$, and its partial derivative w.r.t to $x_{43}$:

\[
\pderiv[1]{A}{x_{43}}
=
\begin{pmatrix}

\pderiv[1]{a_{11}}{x_{43}} & \pderiv[1]{a_{12}}{x_{43}} & \pderiv[1]{a_{13}}{x_{43}} \\\\
\pderiv[1]{a_{21}}{x_{43}} & \pderiv[1]{a_{22}}{x_{43}} & \pderiv[1]{a_{23}}{x_{43}} \\\\
\pderiv[1]{a_{31}}{x_{43}} & \pderiv[1]{a_{32}}{x_{43}} & \pderiv[1]{a_{33}}{x_{43}}

\end{pmatrix}
\]

\[
\pderiv[1]{A}{x_{43}}
= P^T\pderiv[1]{\tilde{X}}{x_{43}} =
\begin{pmatrix}

p_{11} & p_{21} & p_{31} & p_{41} \\
p_{12} & p_{22} & p_{32} & p_{42} \\
p_{13} & p_{23} & p_{33} & p_{43}

\end{pmatrix}
\begin{pmatrix}
0 & 0 & -\frac{1}{4} \\
0 & 0 & -\frac{1}{4} \\
0 & 0 & -\frac{1}{4} \\
0 & 0 & \frac{3}{4}
\end{pmatrix}
=
\begin{pmatrix}

0 & 0 & -\frac{1}{4}p_{11}-\frac{1}{4}p_{21}-\frac{1}{4}p_{31}+\frac{3}{4}p_{41} \\\\
0 & 0 & -\frac{1}{4}p_{12}-\frac{1}{4}p_{22}-\frac{1}{4}p_{32}+\frac{3}{4}p_{42} \\\\
0 & 0 & -\frac{1}{4}p_{13}-\frac{1}{4}p_{23}-\frac{1}{4}p_{33}+\frac{3}{4}p_{43} 

\end{pmatrix}
\]

Performing SVD on $A$, we have $A=USV^T$. Taking the partial derivatives of $A$ w.r.t the element, e.g. $a_{23}$, we have $\pderiv[1]{U}{a_{23}}$ and $\pderiv[1]{V^T}{a_{23}}$. Assuming $d=1$, we can compute $\pderiv[1]{R}{a_{23}}$ as follows: 

\[
\pderiv[1]{R}{a_{23}} = \pderiv[1]{U}{a_{23}}V^T + U\pderiv[1]{V^T}{a_{23}} =
\begin{pmatrix}

\pderiv[1]{r_{11}}{a_{23}} & \pderiv[1]{r_{12}}{a_{23}} & \pderiv[1]{r_{13}}{a_{23}} \\\\
\pderiv[1]{r_{21}}{a_{23}} & \pderiv[1]{r_{22}}{a_{23}} & \pderiv[1]{r_{23}}{a_{23}} \\\\
\pderiv[1]{r_{31}}{a_{23}} & \pderiv[1]{r_{32}}{a_{23}} & \pderiv[1]{r_{33}}{a_{23}}

\end{pmatrix}
\]

Given $\pderiv[1]{a_{23}}{x_{43}} = -\frac{1}{4}p_{12}-\frac{1}{4}p_{22}-\frac{1}{4}p_{32}+\frac{3}{4}p_{42}$ from the above calculation,

\[
\pderiv[1]{R}{x_{43}}
= \pderiv[1]{a_{23}}{x_{43}} \times
\begin{pmatrix}

\pderiv[1]{r_{11}}{a_{23}} & \pderiv[1]{r_{12}}{a_{23}} & \pderiv[1]{r_{13}}{a_{23}} \\\\
\pderiv[1]{r_{21}}{a_{23}} & \pderiv[1]{r_{22}}{a_{23}} & \pderiv[1]{r_{23}}{a_{23}} \\\\
\pderiv[1]{r_{31}}{a_{23}} & \pderiv[1]{r_{32}}{a_{23}} & \pderiv[1]{r_{33}}{a_{23}}

\end{pmatrix}
=
\begin{pmatrix}

\pderiv[1]{r_{11}}{x_{43}} & \pderiv[1]{r_{12}}{x_{43}} & \pderiv[1]{r_{13}}{x_{43}} \\\\
\pderiv[1]{r_{21}}{x_{43}} & \pderiv[1]{r_{22}}{x_{43}} & \pderiv[1]{r_{23}}{x_{43}} \\\\
\pderiv[1]{r_{31}}{x_{43}} & \pderiv[1]{r_{32}}{x_{43}} & \pderiv[1]{r_{33}}{x_{43}}

\end{pmatrix}
\]
 

\end{document}